% Template for Elsevier submission with R Markdown

% Stuff changed from PLOS Template
\documentclass[authoryear, review]{elsarticle}
\usepackage[american]{babel}
\usepackage[section]{placeins}

\bibliographystyle{model5-names}

\journal{Cognitive Psychology}



% amsmath package, useful for mathematical formulas
\usepackage{amsmath}
% amssymb package, useful for mathematical symbols
\usepackage{amssymb}

% hyperref package, useful for hyperlinks
\usepackage{hyperref}

% graphicx package, useful for including eps and pdf graphics
% include graphics with the command \includegraphics
\usepackage{graphicx}

% Sweave(-like)
\usepackage{fancyvrb}
\DefineVerbatimEnvironment{Sinput}{Verbatim}{fontshape=sl}
\DefineVerbatimEnvironment{Soutput}{Verbatim}{}
\DefineVerbatimEnvironment{Scode}{Verbatim}{fontshape=sl}
\newenvironment{Schunk}{}{}
\DefineVerbatimEnvironment{Code}{Verbatim}{}
\DefineVerbatimEnvironment{CodeInput}{Verbatim}{fontshape=sl}
\DefineVerbatimEnvironment{CodeOutput}{Verbatim}{}
\newenvironment{CodeChunk}{}{}

% cite package, to clean up citations in the main text. Do not remove.
\usepackage{cite}

\usepackage{color}

% Use doublespacing - comment out for single spacing
%\usepackage{setspace}
%\doublespacing

% % Text layout
\topmargin 0.0cm
\oddsidemargin 0.5cm
\evensidemargin 0.5cm
\textwidth 15cm
\textheight 21cm

% Bold the 'Figure #' in the caption and separate it with a period
% Captions will be left justified
\usepackage[labelfont=bf,labelsep=period,justification=raggedright]{caption}


% Remove brackets from numbering in List of References
\makeatletter
\renewcommand{\@biblabel}[1]{\quad#1.}
\makeatother


% Leave date blank
\date{}

\begin{document}

\begin{frontmatter}

\title{Social cues modulate the representations underlying cross-situational
learning}

\author[km]{\corref{cor}Kyle MacDonald}
\cortext[cor]{Corresponding author}
\ead{kyle.macdonald@stanford.edu}
\author[dy]{Daniel Yurovsky}
\author[mcf]{Michael C. Frank}
\address{Department of Psychology, Stanford University, United States}


\begin{abstract}
Because children hear language in environments that contain many things
to talk about, learning the meaning of even the simplest word requires
making inferences under uncertainty. A cross-situational statistical
learner can aggregate across naming events to form stable word-referent
mappings, but this approach neglects an important source of information
that can reduce referential uncertainty: social cues from speakers
(e.g., eye gaze). In four large-scale experiments with adults, we tested
the effects of varying referential uncertainty in cross-situational word
learning using social cues. Social cues shifted learners away from
tracking multiple hypotheses and towards storing only a single
hypothesis (Experiments 1 and 2). In addition, learners were sensitive
to graded changes in the strength of a social cue, and when it became
less reliable, they were more likely to store multiple hypotheses
(Experiment 3). Finally, learners stored fewer word-referent mappings in
the presence of a social cue even when visual inspection time was
equivalent to naming events without a social cue present (Experiment 4).
Taken together, our data suggest that the representations underlying
cross-situational word learning are quite flexible: In conditions of
greater uncertainty, learners store a broader range of information.
\end{abstract}

\begin{keyword}
statistical learning, social cues, word learning, language acquisition
\end{keyword}

\end{frontmatter}

\newpage

\section{Introduction}\label{introduction}

Learning the meaning of a new word should be hard. Consider that even
concrete nouns are often used in complex contexts with multiple possible
referents, which in turn have many conceptually natural properties that
a speaker could talk about. This ambiguity creates the potential for an
(in principle) unlimited amount of referential uncertainty in the
learning task.\footnote{This problem is a simplified version of Quine's
  \textit{indeterminacy of reference} (Quine, 1960): That there are many
  possible meanings for a word (``Gavigai'') that include the referent
  (``Rabbit'') in their extension, e.g., ``white,'' ``rabbit,''
  ``dinner.'' Quine's broader philosophical point was that different
  meanings (``rabbit'' and ``undetached rabbit parts'') could actually
  be extensionally identical and thus impossible to tease apart.}
Remarkably, word learning proceeds despite this uncertainty, with
estimates of adult vocabularies ranging between 50,000 to 100,000
distinct words (P. Bloom, 2002). How do learners infer and retain such a
large variety of word meanings from data with this kind of ambiguity?

Statistical learning theories offer a solution to this problem by
aggregating cross-situational statistics across labeling events to
identify underlying word meanings (Siskind, 1996; Yu \& Smith, 2007).
Recent experimental work has shown that both adults and young infants
can use word-object co-occurrence statistics to learn words from
individually ambiguous naming events (Smith \& Yu, 2008; Vouloumanos,
2008). For example, Smith and Yu (2008) taught 12-month-olds three novel
words simply by repeating consistent novel word-object pairings across
10 ambiguous exposure trials. Moreover, computational models suggest
that cross-situational learning can scale up to learn adult-sized
lexicons, even under conditions of considerable referential uncertainty
(K. Smith, Smith, \& Blythe, 2011).

Although all cross-situational learning models agree that the input is
the co-occurrence between words and objects and the output is stable
word-object mappings, they disagree about how closely learners
approximate the input distribution (for review, see Smith, Suanda, \& Yu
2014). One approach has been to model learning as a process of updating
connection strengths between multiple word-object links (McMurray,
Horst, \& Samuelson, 2012), while other approaches have argued that
learners store only a single word-object hypothesis (Trueswell, Medina,
Hafri, \& Gleitman, 2013). In recent experimental and modeling work
Yurovsky and Frank (2015) suggest an integrative explanation: learners
allocate a fixed amount of attention to a single hypothesis, and
distribute the rest evenly among the remaining alternatives. As the set
of alternatives grows, the amount of attention allocated to each object
approaches zero.

In addition to the debate about representation, researchers have
disagreed about how to characterize the ambiguity of the input to
cross-situational learning mechanisms. One way to quantify the
uncertainty in a naming event is to show adults clips of caregiver-child
interactions and measure their accuracy at guessing the meaning of an
intended referent (Human Simulation Paradigm: HSP {[}Gillette, Gleitman,
Gleitman, and Lederer, 1999{]}). Using the HSP, Medina, Snedeker,
Trueswell, and Gleitman (2011) found that approximately 90\% of learning
episodes were ambiguous (\textless{} 33\% accuracy) and only 7\% were
relatively unambiguous (\textgreater{} 50\% accuracy). In contrast,
Yurovsky, Smith, and Yu (2013) found a higher proportion of clear naming
events, with approximately 30\% being unambiguous (\textgreater{} 90\%
accuracy). Consistent with this finding, Cartmill, Armstrong, Gleitman,
Goldin-Meadow, Medina, and Trueswell (2013) showed that the proportion
of unambiguous naming episodes varies across parent-child dyads, with
some parents rarely providing highly informative contexts and others'
doing so relatively more often.\footnote{The differences in the
  estimates of referential uncertainty in these studies could be driven
  by the different sampling procedures used to select naming events for
  the HSP. Yurovsky, Smith, and Yu (2013) sampled utterances for which
  the parent labeled a co-present object, whereas Medina, Snedeker,
  Trueswell, et al. (2011) randomly sampled any utterances containing
  concrete nouns. Regardless of these differences, the key point here is
  that variability in referential uncertainty across naming events
  exists and thus could alter the representations underlying
  cross-situational learning.}

Thus, representations in cross-situational word learning can appear
distributional or discrete, and the input to statistical learning
mechanisms can vary along a continuum from low to high ambiguity. These
results raise an interesting question: could learners be sensitive to
the ambiguity of the input and use this information to alter the
representations they store in memory? In the current line of work, we
investigated how the presence of referential cues in the social context
might alter the ambiguity of the input to statistical word learning
mechanisms.

Social-pragmatic theories of language acquisition emphasize the
importance of social cues for word learning (P. Bloom, 2002; Clark,
2009; Hollich et al., 2000). Experimental work has shown that even
children as young as 16 months prefer to map novel words to objects that
are the target of a speaker's gaze and not their own (Baldwin, 1993). In
an analysis of naturalistic parent-child labeling events, Yu and Smith
(2012) found that young learners tended to retain labels that were
accompanied by clear referential cues, which served to make a single
object dominant in the visual field. And correlational studies have
demonstrated strong links between early intention-reading skills (e.g.,
gaze following) and later vocabulary growth (Brooks \& Meltzoff, 2005,
2008; Carpenter, Nagell, Tomasello, Butterworth, \& Moore, 1998).
Moreover, studies outside the domain of language acquisition have shown
that the presence of social cues: (a) produce better spatial learning of
audiovisual events (Wu, Gopnik, Richardson, \& Kirkham, 2011), (b) boost
recognition of a cued object (Cleveland, Schug, \& Striano, 2007), and
(c) lead to preferential encoding of an object's featural information
(Yoon, Johnson, \& Csibra, 2008). Together, the evidence suggests that
social cues could alter the representations stored during
cross-situational word learning by modulating how people allocate
attention to the relevant statistics in the input.

The goal of our current investigation is to ask whether the presence of
a valid social cue -- a speaker's gaze -- can change the representations
underlying cross-situational word learning. We used a modified version
of Yurovsky and Frank (2015)'s paradigm to provide a direct measure of
memory for alternative word-object links during cross-situational
learning. In Experiment 1, we manipulated the presence of a referential
cue at different levels of attention and memory demands. At all levels
of difficulty, learners tracked a strong single hypothesis but were less
likely to track multiple word-object links when a social cue was
present. In Experiment 2, we replicated the findings from Experiment 1
using a more ecologically valid social cue. In Experiment 3, we moved to
a parametric manipulation of referential uncertainty by varying the
reliability of the speaker's gaze. Learners were sensitive to graded
changes in reliability and retained more word-object links as
uncertainty in the input increased. Finally, in Experiment 4, we equated
the length of the initial naming events with and without the referential
cue. Learners stored less information in the presence of gaze even when
they had visually inspected the objects for the same amount of time. In
sum, our data suggest that cross-situational word learners are quite
flexible, storing representations with different levels of fidelity
depending on the amount of ambiguity present during learning.

\section{Experiment 1}\label{experiment-1}

We set out to test the effect of a referential cue on the
representations underlying cross-situational word learning. We used a
version of Yurovsky and Frank (2015)'s paradigm where we manipulated the
ambiguity of the learning context by including a gaze cue from a
schematic, female interlocutor. Participants saw a series of ambiguous
exposure trials where they heard one novel word that was either paired
with a gaze cue or not and selected the object they thought went with
each word. In subsequent test trials, participants heard the novel word
again, this time paired with a new set of novel objects. One of the
objects in this set was either the participant's initial guess (Same
test trials) or one of the objects was \emph{not} their initial guess
(Switch test trials). Performance on Switch trials provided a direct
measure of whether referential cues influenced the number of alternative
word-object links that learners stored in memory. If learners performed
worse on Switch trials after an exposure trial with gaze, this would
suggest that they stored fewer additional objects from the initial
learning context.

\subsection{Method}\label{method}

\subsubsection{Participants}\label{participants}

We posted a set of Human Intelligence Tasks (HITs) to Amazon Mechanical
Turk. Only participants with US IP addresses and a task approval rate
above 95\% were allowed to participate, and each HIT paid 30 cents.
50-100 HITs were posted for each of the 32 between-subjects conditions.
Data were excluded if participants completed the task more than once or
if participants did not respond correctly on familiar object trials (131
HITs). The final sample consisted of 1438 participants.

\begin{CodeChunk}
\begin{figure}[tb]

{\centering \includegraphics[width=0.8\linewidth]{figs/stimuli-1} 

}

\caption[Screenshots of exposure and test trials from Experiment 1 (schematic gaze cue) and Experiments 2, 3 \& 4 (video gaze cue)]{Screenshots of exposure and test trials from Experiment 1 (schematic gaze cue) and Experiments 2, 3 \& 4 (video gaze cue). Participants saw exposure trials with or without a gaze cue depending on condition assignment. All participants saw both types of test trials: Same and Switch. On Same trials, the object that participants chose during exposure appeared with a new novel object. On Switch trials the object that participants did not choose appeared with a new novel object.}\label{fig:stimuli}
\end{figure}
\end{CodeChunk}

\subsubsection{Stimuli}\label{stimuli}

Figure 1 shows screenshots taken from Experiment 1. Visual stimuli were
black and white pictures of familiar and novel objects taken from
Kanwisher, Woods, Iacoboni, and Mazziotta (1997). Auditory stimuli were
recordings of familiar and novel words by an AT\&T Natural Voices
\texttrademark (voice: Crystal) speech synthesizer. Novel words were 1-3
syllable pseudowords that obeyed all rules of English phonotactics. A
schematic drawing of a human speaker was chosen for ease of manipulating
the direction of gaze, the referential cue of interest in this study.
All experiments can be viewed and downloaded at the project page:
\url{https://kemacdonald.github.io/soc_xsit/}.

\subsubsection{Design and Procedure}\label{design-and-procedure}

Participants saw a total of 16 trials: eight exposure trials and eight
test trials. On each trial, they heard one novel word, saw a set of
novel objects, and were asked to guess which object went with the word.
Before seeing exposure and test trials, participants completed four
practice trials with familiar words and objects. These trials
familiarized participants to the task and allowed us to exclude
participants who were unlikely to perform the task as directed, either
because of inattention or because their computer audio was turned off.

After the practice trials, participants were told that they would now
hear novel words and see novel objects and that their task was to select
the referent that ``goes with each word.'' Over the course of the
experiment, participants heard eight novel words two times, with one
exposure trial and one test trial for each word. Four of the test trials
were \emph{Same} trials in which the object that participants selected
on the exposure trial was shown with a set of new novel objects. The
other four test trials were \emph{Switch} trials in which one of the
objects was chosen at random from the set of objects that the
participant did not select on exposure.

Participants were randomly assigned to one of the 32 between-subjects
conditions (4 Referents X 4 Intervals X 2 Gaze conditions). Participants
either saw 2, 4, 6, or 8 referents on the screen and test trials
occurred at different intervals after exposure trials: either 0, 1, 3,
or 7 trials from the initial exposure to a word. For example, in the
0-interval condition, the test trial for that word would occur
immediately following the exposure trial, but in the 3-interval
condition, participants would see three additional exposure trials for
other novel words before seeing the test trial for the initial word. The
interval conditions modulated the time delay and the number of
intervening trials between learning and test, and the number of
referents conditions modulated the attention demands present during
learning.

Participants were assigned to either the Gaze or No-Gaze condition. In
the Gaze condition, gaze was directed towards one of the objects on
exposure trials; in the No-Gaze condition, gaze was always directed
straight ahead (see Figure 1 for examples). At test, gaze was always
directed straight ahead. To show participants that their response had
been recorded, a red box appeared around the selected object for one
second. This box always appeared around the selected object, even if
participants' selections were incorrect.

\subsection{Results and Discussion}\label{results-and-discussion}

\subsubsection{Analysis plan}\label{analysis-plan}

The structure of our analysis plan is parallel across all four
experiments. First, we examined accuracy and response time on exposure
trials to provide evidence that learners were (a) sensitive to our
experimental manipulation and (b) altered their allocation of attention
in response to the presence of a social cue. Accuracy on exposure trials
was defined as selecting the referent that was the target of gaze in the
Gaze condition. (Note that there was no ``correct'' behavior for
exposure trials in the No-Gaze condition.) Next, we examined accuracy on
test trials to test whether learners' memory for alternative word-object
links changed depending on the ambiguity of the learning context.
Accuracy on test trials (both Same and Switch) was defined as selecting
the referent that was present during the exposure trial for that word.

The key behavioral prediction of our hypothesis is that the presence of
gaze would result in reduced memory for multiple word-object links,
operationalized as a decrease in accuracy on Switch test trials after
seeing exposure trials with a gaze cue. To quantify participants'
behavior, we used mixed-effects regression models with the maximal
random effects structure justified by our experimental design:
by-subject intercepts and slopes for each trial type (Barr, 2013). We
limited all models to include only two-way interactions because the
critical test of our hypothesis was the interaction between gaze
condition and trial type, and we did not have theoretical predictions
for any possible three-way or four-way interactions. All models were fit
using the lme4 package in R (Bates, Maechler, Bolker, \& Walker, 2013),
and all of our data and our processing/analysis code can be viewed in
the version control repository for this paper at
\url{https://github.com/kemacdonald/soc_xsit}.

\begin{CodeChunk}
\begin{figure}[tb]

{\centering \includegraphics[width=0.9\linewidth]{figs/expt1-plot-1} 

}

\caption[Experiment 1 results]{Experiment 1 results. The top row shows average inspection times on exposure trials for all experimental conditions as a function of the number of trials that occurred between exposure and test. Each panel represents a different number of referents, and line color represents the Gaze and No-Gaze conditions. The bottom row shows accuracy on test trials for all conditions as a function of the number of intervening trials. The horizontal dashed lines represent chance performance for each number of referents, and the type of line (solid vs. dashed) represents the different test trial types (Same vs. Switch). Error bars indicate 95\% confidence intervals computed by non-parametric bootstrap.}\label{fig:expt1-plot}
\end{figure}
\end{CodeChunk}

\subsubsection{Exposure trials}\label{exposure-trials}

To ensure that our referential cue manipulation was effective, we
compared participants' accuracies on exposure trials in the Gaze
condition to a model of random behavior defined as a Binomial
distribution with a probability of success \(\frac{1}{Num Referents}\).
Correct performance was defined as selecting the object that was the
target of the speaker's gaze. Following Yurovsky and Frank (2015), we
fit logistic regressions for each gaze, referent, and interval
combination specified as
\texttt{Gaze Target $\sim$ 1 + offset(logit(1/Referents))}. The offset
encoded the chance probability of success given the number of referents,
and the coefficient for the intercept term shows on a log-odds scale how
much more likely participants were to select the gaze target than would
be expected if participants were selecting randomly. In all conditions,
participants used gaze to select referents on exposure trials more often
than expected by chance (smallest \(\beta\) = 1.4, z = 9.38, \(p\)
\textless{} .001). However, there was variability across conditions in
the mean proportion of gaze cue (overall \(M\) = 0.84, range:
0.77--0.93).

We were also interested in differences in participants' response times
across the experimental conditions. Since these trials were self-paced,
participants could choose how much time to spend inspecting the
referents on the screen, thus providing an index of participants'
attention. To quantify the effects of gaze, interval, and number of
referents, we fit a linear mixed-effects model that predicted
participants' inspection times as follows:
\texttt{Log(Inspection time) $\sim$ Gaze * Log(Interval) + Gaze * Log(Referents) + (1 | subject)}.
We found a significant main effect of the number of referents (\(\beta\)
= 0.34, p \textless{} .001) with longer inspection times as the number
of referents increased, a significant interaction between gaze condition
and the number of referents (\(\beta\) = -0.27, p \textless{} .001) with
longer inspection times in the No-Gaze condition, especially as the
number of referents increased, and a significant interaction between
gaze condition and interval (\(\beta\) = -0.08, \(p\) = 0.004) with
slower inspection times in the No-Gaze condition, especially as the
number of intervening trials increased (see the top row of Figure 2).
Shorter inspection times on exposure trials with gaze provide evidence
that the presence of a referential cue focused participants' attention
on a single referent and away from alternative word-object links.

\subsubsection{Test trials}\label{test-trials}

Next, we explored participants' accuracy in identifying the referent for
each word in all conditions for both kinds of test trials (see the
bottom row of Figure 2). We first compared the distribution of correct
responses made by each participant to the distribution expected if
participants were selecting randomly defined as a Binomial distribution
with a probability of success \(\frac{1}{Num Referents}\). Correct
performance was defined as selecting the object that was present on the
exposure trial for that word. We fit the same logistic regressions as we
did for exposure trials:
\texttt{Correct $\sim$ 1 + offset(logit(1/Referents))}. In 31 out of the
32 conditions for both Same and Switch trials, participants chose the
correct object more often than would be expected by chance (smallest
\(\beta\) = 0.36, \(z\) = 2.44, \(p\) = 0.01). On Switch trials in the
8-referent, 3-interval condition, participants' responses were not
significantly different from chance (\(\beta\) = 0.06, z = 0.33, \(p\) =
0.74). Participants' success on Switch trials replicates the findings
from Yurovsky and Frank (2015) and provides direct evidence that
learners encoded more than a single hypothesis in ambiguous word
learning situations even under high attentional and memory demands and
in the presence of a referential cue.

\begin{table}[tb]
\centering
\begin{tabular}{lrrrrl}
 Predictor & Estimate & Std. Error & $z$ value & $p$ value &  \\ 
  \hline
Intercept & 3.01 & 0.29 & 10.35 & $<$ .001 & *** \\ 
  Switch Trial & -1.36 & 0.24 & -5.63 & $<$ .001 & *** \\ 
  Gaze Condition & 0.12 & 0.26 & 0.47 & 0.64 &  \\ 
  Log(Interval) & -0.45 & 0.11 & -4.08 & $<$ .001 & *** \\ 
  Log(Referents) & 0.23 & 0.11 & 2.02 & 0.04 & * \\ 
  Switch Trial*Gaze Condition & -1.09 & 0.12 & -9.07 & $<$ .001 & *** \\ 
  Switch Trial*Log(Interval) & 0.52 & 0.05 & 9.50 & $<$ .001 & *** \\ 
  Switch Trial*Log(Referent) & -0.59 & 0.09 & -6.49 & $<$ .001 & *** \\ 
  Gaze Condition*Log(Interval) & 0.06 & 0.06 & 1.00 & 0.32 &  \\ 
  Gaze Condition*Log(Referent) & 0.20 & 0.09 & 2.15 & 0.03 & * \\ 
  Log(Interval)*Log(Referent) & -0.04 & 0.04 & -1.02 & 0.31 &  \\ 
   \hline
\end{tabular}
\caption{Predictor estimates with standard errors and significance information for a logistic mixed-effects model predicting word learning in Experiment 1.} 
\label{tab:exp1_reg}
\end{table}

To quantify the effects of gaze, interval, and number of referents on
the probability of a correct response, we fit the following
mixed-effects logistic regression model to a filtered dataset where we
removed participants who did not reliably select the object that was the
target of gaze on exposure trials:\footnote{We did not predict that
  there would be a subset of participants who would not follow the gaze
  cue, thus this filtering criteria was developed posthoc. However, we
  think that the filter is theoretically motivated because we would only
  expect to see an effect of gaze if participants actually used the gaze
  cue. The filter removed 94 participants (6\% of the sample). The key
  inferences from the data do not depend on this filtering criteria.}
\texttt{Correct $\sim$ Trial Type * Gaze + Trial Type * Log(Interval) + Trial Type * \\ Log(Referents) + offset(logit($^1/_{Referents}$)) + (TrialType | subject)}.

We coded interval and number of referents as continuous predictors and
transformed these variables to the log scale. We limited the model to
include only two-way interactions because the critical test of our
hypothesis is the interaction between gaze condition and trial type, and
we did not have any theoretical predictions for possible three-way
interactions.
\footnote{If we allowed for three-way interactions in the model, there was a marginally significant interaction between gaze condition, trial type, and interval ($\beta = 0.21$, $p$ = 0.058). The two-way interaction between gaze condition and trial type remained significant in this more complex model ($\beta = -1.3$, $p$ = 0.006).}

Table 1 shows the output of the logistic regression. We found
significant main effects of the number of referents (\(\beta = 0.23\),
\(p\) \textless{} .001) and interval (\(\beta = -0.45\), \(p\)
\textless{} .001), such that as each of these factors increased,
accuracy on test trials decreased. We also found a significant main
effect of trial type (\(\beta = -1.36\), \(p\) \textless{} .001), with
worse performance on Switch trials. There were significant interactions
between trial type and interval (\(\beta = 0.52\), \(p\) \textless{}
.001), trial type and referents (\(\beta = -0.59\), \(p\) \textless{}
.001), and gaze condition and referents (\(\beta = 0.2\), \(p\)
\textless{} .05). These interactions can be interpreted as meaning: (a)
the interval between exposure and test affected Same trials more than
Switch trials, (b) the number of referents affected Switch trials more
than Same trials, and (c) participants performed slightly better at the
higher number of referents in the Gaze condition. The interactions
between gaze condition and referents and between referents and interval
were not significant. Importantly, we found the predicted interaction
between trial type and gaze condition (\(\beta = -1.09\), \(p\)
\textless{} .001), with participants in the Gaze condition performing
worse on Switch trials. This interaction provides direct evidence that
the presence of a referential cue reduces participants' memory for
alternative word-object links.

We were also interested in how the length of inspection times on
exposure trials would affect participants' accuracy at test. So we fit
an additional model where participants' inspection times were included
as a predictor. We found a significant interaction between inspection
time and gaze condition (\(\beta = -0.17\), \(p\) = 0.01), such that
longer inspection times provided a larger boost to accuracy in the
No-Gaze condition. Importantly, the key test of our hypothesis, the
interaction between gaze condition and trial type, remained significant
in this alternative version of the model (\(\beta\) = -1.02, \(p\) = p
\textless{} .001).

Taken together, the inspection time and accuracy analyses provide
evidence that the presence of a referential cue modulated learners'
attention during learning, and in turn made them less likely to track
multiple word-object links. We saw some evidence for a boost to
performance on Same trials in the Gaze condition at the higher number of
referent and interval conditions, but reduced tracking of alternatives
did not always result in better memory for learners' candidate
hypothesis. This finding suggests that the limitations on Same trials
may be different than those regulating the distribution of attention on
Switch trials.

There was relatively large variation in performance across conditions in
the group-level accuracy scores and in participants' tendency to
\emph{use} the referential cue on exposure trials. Moreover, we found a
subset of participants who did not reliably use the gaze cue at all,
potentially reducing the effect of gaze on cross-situational learning in
this experiment. It is possible that the effect of gaze was reduced
because the referential cue that we used -- a static schematic drawing
of a speaker -- was relatively weak compared to the cues present in
real-world learning environments. Thus we do not yet know how learners'
memory for alternatives during cross-situational learning would change
in the presence of a stronger and more ecologically valid referential
cue. We designed Experiment 2 to address this question.

\section{Experiment 2}\label{experiment-2}

In Experiment 2, we set out to replicate the findings from Experiment 1
using a more ecologically valid stimulus set. We replaced the static,
schematic drawing with a video of a female actress. While these stimuli
were still far from actual learning contexts, they included a real
person who provided both a gaze cue and a head turn towards the target
object. To reduce the across-conditions variability that we found in
Experiment 1, we introduced a within-subjects design where each
participant saw both Gaze and No-Gaze exposure trials in a blocked
design. We selected a subset of the conditions from Experiment 1 and
tested only the 4-referent display with 0 and 3 intervening trials as
between-subjects manipulations. Our goals were to replicate the
reduction in learners' tracking of alternative word-object links in the
presence of a referential cue and to test whether increasing the
ecological validity of the cue would result in a boost to the strength
of learners' recall of their candidate hypothesis.

\subsection{Method}\label{method-1}

\subsubsection{Participants}\label{participants-1}

Participant recruitment and inclusion/exclusion criteria were identical
to those of Experiment 1. 100 HITs were posted for each condition (1
Referent X 2 Intervals X 2 Gaze conditions) for total of 400 paid HITs
(33 HITs excluded).

\subsubsection{Stimuli}\label{stimuli-1}

Audio and picture stimuli were identical to Experiment 1. The
referential cue in the Gaze condition was a video (see Figure 1). On
each exposure trial, the actress looked out at the participant with a
neutral expression, smiled, and then turned to look at one of the four
images on the screen. She maintained her gaze for 3 seconds before
returning to the center. On test trials, she looked straight ahead for
the duration of the trial.

\subsection{Design and Procedure}\label{design-and-procedure-1}

Procedures were identical to those of Experiment 1. The major design
change was a within-subjects manipulation of the gaze cue where each
participant saw exposure trials with and without gaze. The experiment
consisted of 32 trials split into 2 blocks of 16 trials. Each block
consisted of 8 exposure trials and 8 test trials (4 Same trials and 4
Switch trials) and contained only Gaze or No-gaze exposure trials. The
order of block was counterbalanced across participants.

\subsection{Results and Discussion}\label{results-and-discussion-1}

We followed the same analysis plan as in Experiment 1. We first analyzed
inspection times and accuracy on exposure trials and then analyzed
accuracy on test trials.

\subsubsection{Exposure trials}\label{exposure-trials-1}

Similar to Experiment 1, participants' responses on exposure trials
differed from those expected by chance (smallest \(\beta\) = 3.39, z =
31.99, \(p\) \textless{} .001), suggesting that gaze was effective in
directing participants' attention. Participants in Experiment 2 were
more consistent in their use of gaze with the video stimuli compared to
the schematic stimuli used in Experiment 1 (\(M_{Exp1}\) = 0.8,
\(M_{Exp2}\) = 0.91\$), suggesting that using a real person increased
participants' willingness to follow the gaze cue.

We replicated the findings from Experiment 1. Inspection times were
shorter in the Gaze (\(\beta\) = -1.1, \(p\) \textless{} .001) and the
3-interval condition (\(\beta\) = -0.48, \(p\) \textless{} .001). The
interaction between gaze and interval was not significant, meaning that
gaze had the same effect on participants' inspection times at both
intervals (see Panel A of Figure 3).

\begin{CodeChunk}
\begin{figure}[tb]

{\centering \includegraphics[width=0.8\linewidth]{figs/expt2-plot-1} 

}

\caption[Experiment 2 results]{Experiment 2 results. Panel A shows inspection times on exposure trials with and without gaze. Panel B shows accuracy on Same and Switch test trials. All plotting conventions are the same as in Figure 2. Error bars indicate 95\% confidence intervals computed by non-parametric bootstrap.}\label{fig:expt2-plot}
\end{figure}
\end{CodeChunk}

\subsubsection{Test trials}\label{test-trials-1}

\begin{table}[tb]
\centering
\begin{tabular}{lrrrrl}
 Predictor & Estimate & Std. Error & $z$ value & $p$ value &  \\ 
  \hline
Intercept & 2.94 & 0.18 & 16.00 & $<$ .001 & *** \\ 
  Switch Trial & -2.99 & 0.19 & -16.11 & $<$ .001 & *** \\ 
  Gaze Condition & -0.10 & 0.16 & -0.63 & 0.53 &  \\ 
  Log(Interval) & -0.93 & 0.10 & -9.23 & $<$ .001 & *** \\ 
  Switch Trial*Gaze Condition & -0.71 & 0.16 & -4.49 & $<$ .001 & *** \\ 
  Switch Trial*Log(Interval) & 0.79 & 0.10 & 8.03 & $<$ .001 & *** \\ 
  Gaze Condition*Log(Interval) & 0.15 & 0.08 & 2.05 & 0.04 & * \\ 
   \hline
\end{tabular}
\caption{Predictor estimates with standard errors and significance information for a logistic mixed-effects model predicting word learning in Experiment 2.} 
\label{tab:exp2_reg}
\end{table}

Across all conditions for both trial types, participants selected the
correct referent at rates greater than chance (smallest \(\beta\) =
0.58, z = 9.32, \(p\) \textless{} .001). We replicated the critical
finding from Experiment 1: after seeing exposure trials with gaze,
participants performed worse on Switch trials, meaning they stored fewer
word-object links (\(\beta = -0.71\), \(p\) \textless{}
.001).\footnote{As in Experiment 1, we fit this model to a filtered dataset removing participants who did not reliably use the gaze cue.}
Participants were also less accurate as the interval between exposure
and test increased (\(\beta\) = -0.93, \(p\) \textless{} .001) and on
the Switch trials overall (\(\beta = -2.99\), \(p\) \textless{} .001).

In addition, there was a significant interaction between trial type and
interval (\(\beta = 0.79\), \(p\) \textless{} .001), with worse
performance on Switch trials in the 3-interval condition. The
interaction between gaze condition and interval was also significant
(\(\beta = 0.15\), \(p\) = 0.041), such that participants in the gaze
condition were less affected by the increase in interval. Similar to
Experiment 1, we did not see evidence of a boost to performance on Same
trials in the gaze condition.

Next, we added inspection times on exposure trials to the model. Similar
to Experiment 1, the key interaction between gaze and trial type
remained significant in this version of the model (\(\beta\) = -0.54,
\(p\) \textless{} .001). However, we found an interaction between
inspection time and trial type (\(\beta\) = 0.21, \(p\) = 0.05), with
longer inspection times providing a larger boost to performance on
Switch trials. This result differs slightly from what we found in
Experiment 1 where longer inspection times led to better accuracy in the
No-Gaze condition. It seems plausible that more time spent visually
exploring the objects during learning would lead to better performance
on Switch trials, which depend on encoding multiple alternatives. Thus,
the interaction between inspection time and gaze condition found in
Experiment 1 might have been driven by the fact that longer inspection
times were more likely to occur in the absence of a gaze cue.

The results of Experiment 2 provide converging evidence for our primary
hypothesis that the presence of a referential cue reliably focuses
learners' attention away from alternative word-object links and shifts
them towards single hypothesis tracking. Moving to the video stimulus
led to higher rates of selecting the target of gaze on exposure trials,
but did not result in a boost to performance on Same trials. This
finding suggests that the level of attention and memory demand present
in the learning context might modulate the effect of gaze on the
fidelity of learners' single hypothesis.

Thus far we have shown that people store different amounts of
information in response to a categorical manipulation of referential
uncertainty. In both Experiments 1 and 2, the learning context was
either entirely ambiguous (No-Gaze) or entirely unambiguous (Gaze). But
not all real-world learning contexts fall at the extremes of this
continuum. Could learners be sensitive to more subtle changes in the
quality of the input? In our next experiment, we tested a prediction of
our account: whether learners would store more word-object links in
response to graded changes in referential uncertainty during learning.

\section{Experiment 3}\label{experiment-3}

In Experiment 3, we explored whether learners would allocate attention
and memory flexibly in response to \emph{graded} changes in the
referential uncertainty that was present during learning. To test this
hypothesis, we moved beyond a categorical manipulation of the
presence/absence of gaze, and we parametrically varied the reliability
of the referential cue. We manipulated cue reliability by adding a block
of familiarization trials where we varied the proportion of Same and
Switch trials. If participants saw more Switch trials, this provided
direct evidence that the speaker's gaze was a less reliable cue to
reference because the gaze target on exposure trials would not appear at
test. This design was inspired by a growing body of experimental work
showing that even young children are sensitive to the prior reliability
of speakers and will use this information to decide whom to learn novel
words from (e.g., Koenig, Clement, \& Harris, 2004).

\subsection{Method}\label{method-2}

\subsubsection{Participants}\label{participants-2}

Participant recruitment and inclusion/exclusion criteria were identical
to those of Experiment 1 and 2 (27 HITs excluded). 100 HITs were posted
for each reliability level (0\%, 25\%, 50\%, 75\%, and 100\%) for total
of 500 paid HITs.

\subsubsection{Design and Procedure}\label{design-and-procedure-2}

Procedures were identical to those of Experiments 1 and 2. We modified
the design of our cross-situational learning paradigm to include a block
of 16 familiarization trials (8 exposure trials and 8 test trials) at
the beginning of the experiment. These trials served to establish the
reliability of the speaker's gaze. To establish reliability, we varied
the proportion of Same/Switch trials that occurred during the
familiarization block. Recall that on Switch trials the gaze target did
not show up at test, which provided evidence that the speaker's gaze was
not a reliable cue to reference. Reliability was a between-subjects
manipulation such that participants either saw 8, 6, 4, 2, or 0 Switch
trials during familiarization, which created the 0\%, 25\%, 50\%, 75\%,
and 100\% reliability conditions. After the familiarization block,
participants completed another block of 16 trials (8 exposure trials and
8 test trials). Since we were no longer testing the effect of the
presence or absence of a referential cue, all exposure trials throughout
the experiment included a gaze cue. Finally, at the end of the task, we
asked participants to assess the reliability of the speaker on a
continuous scale from ``completely unreliable'' to ``completely
reliable.''

\subsection{Results and Discussion}\label{results-and-discussion-2}

\subsubsection{Exposure trials}\label{exposure-trials-2}

Participants reliably chose the referent that was the target of gaze at
rates greater than chance (smallest \(\beta\) = 2.62, z = 31.99, \(p\)
\textless{} .001). We fit a mixed effects logistic regression model
predicting the probability of selecting the gaze target as follows:
\texttt{Correct-Exposure $\sim$ Reliability Condition * Subjective Reliability + (1 | subject)}.
We found an effect of reliability condition (\(\beta\) = 3.28, \(p\) =
0.03) such that when the gaze cue was more reliable, participants were
more likely to use it (\(M_{0\%}\) = 0.83, \(M_{25\%}\) = 0.82,
\(M_{50\%}\) = 0.87, \(M_{75\%}\) = 0.9, \(M_{100\%}\) = 0.94). We also
found an effect of subjective reliability (\(\beta\) = 7.26, \(p\)
\textless{} .001) such that when participants thought the gaze cue was
reliable, they were more likely to use it. The interaction between
reliability condition and subjective reliability assessments was
marginally significant (\(\beta\) = -4.58, \(p\)= 0.092). This analysis
provides evidence that participants were sensitive to the reliability
manipulation both in how often they used the gaze cue and in how they
rated the reliability of the speaker at the end of the task.

\subsubsection{Test trials}\label{test-trials-2}

\begin{CodeChunk}
\begin{figure}[tb]

{\centering \includegraphics[width=0.9\linewidth]{figs/e3-plot-1} 

}

\caption[Primary analyses of test trial performance in Experiment 3]{Primary analyses of test trial performance in Experiment 3. Panel A shows performance as a function of reliability condition. Panel B shows performance as a function of reliability condition and whether participants chose to follow gaze on exposure trials. The horizontal dashed lines represent chance performance, and error bars indicate 95\% confidence intervals computed by non-parametric bootstrap.}\label{fig:e3-plot}
\end{figure}
\end{CodeChunk}

Next, we tested whether the reliability manipulation altered the
strength of participants' memory for alternative word-object links.
Across all conditions, participants selected the correct referent at
rates greater than chance (smallest \(\beta\) = 0.42, z = 3.69, \(p\)
\textless{} .001). Our primary prediction was an interaction between
reliability and test trial type, with higher levels of reliability
leading to worse performance on Switch trials (i.e., less memory
allocated to alternative word-object links). To explore this prediction,
we performed four complementary analyses: our primary analysis, which
tested the effect of the reliability manipulation, and three secondary
analyses, which explored the effects of participants' (a) use of the
gaze cue, (b) subjective reliability assessments, and (c) inspection
time on exposure trials.

\begin{CodeChunk}
\begin{figure}[tb]

{\centering \includegraphics[width=0.9\linewidth]{figs/expt3-sub-plots-1} 

}

\caption[Secondary analyses of test trial performance in Experiment 3]{Secondary analyses of test trial performance in Experiment 3. Panel A shows accuracy as a function of the number of exposure trials on which participants chose to use the gaze cue. Panel B shows accuracy as a function of participants' subjective reliability judgments. The horizontal dashed lines represent chance performance, and error bars indicate 95\% confidence intervals computed by non-parametric bootstrap.}\label{fig:expt3-sub-plots}
\end{figure}
\end{CodeChunk}

\paragraph{Reliability condition
analysis}\label{reliability-condition-analysis}

To test the effect of reliability, we fit a model predicting accuracy at
test using reliability condition and test trial type as predictors. We
found a significant main effect of trial type (\(\beta = -3.95\), \(p\)
\textless{} .001), with lower accuracy on Switch trials. We also found
the key interaction between reliability condition and trial type
(\(\beta\) = -0.76, \(p\) = 0.044), such that when gaze was more
reliable, participants performed worse on Switch trials (see Panel A of
Figure 4). This interaction suggests that people stored more word-object
links as the learning context becomes more ambiguous. However, the
interaction between reliability and trial type was not particularly
strong, and -- similar to Experiment 1 -- there was variability in
performance across conditions (see the 50\% reliable condition in Panel
A of Figure 4). So to provide additional support for our hypothesis, we
conducted three follow-up analyses.

\paragraph{Gaze use analyses}\label{gaze-use-analyses}

We would only expect to see a strong interaction between reliability and
trial type if learners chose to use the gaze cue during exposure trials.
To test this hypothesis, we fit two additional models that included two
different measures of participants' use of the gaze cue. First, we added
accuracy on exposure trials as a predictor in our model. (Recall that
correct performance on exposure trials was defined as using the gaze
cue.) We found a significant interaction between accuracy on exposure
trials and trial type (\(\beta = -1.43\), \(p\) \textless{} .001) with
worse performance on Switch test trials when participants used gaze on
exposure trials (see Panel B of Figure 4). We also found an interaction
between gaze use and reliability (\(\beta = 0.97\), \(p\) = 0.004) such
that when gaze was more reliable, participants were more likely to use
it. The interaction between trial type and reliability became marginally
significant in this model (\(\beta = -0.62\), \(p\) = 0.086), suggesting
that participants' use of the gaze cue was a stronger predictor of
memory for alternative word-object links.\footnote{We are grateful to an
  anonymous reviewer for suggesting this analysis, but we would like to
  note that it is exploratory.}

We also hypothesized that the reliability manipulation might change how
often individual participants chose to use the gaze cue throughout the
task. To explore this possibility, we fit a model with the same
specifications, but we included a predictor that we created by binning
participants based on the number of exposure trials on which they chose
to follow gaze (i.e., a gaze following score). We found a significant
interaction between how often participants chose to follow gaze on
exposure trials and trial type (\(\beta = -0.32\), \(p\) \textless{}
.001), such that participants who were more likely to use the gaze cue
performed worse on Switch trials, but not Same trials (see Panel B of
Figure 5).\footnote{We found this interaction while performing
  exploratory data analysis on a previous version of this study with an
  independent sample (N = 250, \(\beta = -0.29\), \(p\) \textless{}
  .001). The results reported here are from a follow-up study where
  testing this interaction was a planned analysis.} Taken together, the
two analyses of participants' use of the gaze cue provide converging
evidence that when the speaker's gaze was reliable participants were
more likely to use the cue, and when they followed gaze, they tended to
store less information from the initial naming event.

\paragraph{Subjective reliability
analysis}\label{subjective-reliability-analysis}

The strong interaction between use of the gaze cue and memory for
alternative word-object links suggests that participants' subjective
experience of reliability in the experiment mattered. Thus, we fit the
same model but substituted subjective reliability for the frequency of
gaze use as a predictor of test trial performance. We found a
significant interaction between trial type and participants' subjective
reliability assessments (\(\beta = -1.63\), \(p\) = 0.01): when
participants thought the speaker was more reliable, they performed worse
on Switch trials, but not Same trials (see Panel B of Figure 5).

\paragraph{Inspection time analyses}\label{inspection-time-analyses}

Finally, we analyzed the effect of inspection times on exposure trials,
fitting a model using inspection time, trial type, and reliability
condition to predict accuracy at test. We found a main effect of
inspection time (\(\beta\) = 0.31, \(p\) = 0.001), with longer
inspection times leading to better performance for both Same and Switch
trials. There was a marginally significant interaction between
inspection time and reliability condition (\(\beta\) = -0.2, \(p\) =
0.067) with longer inspection times providing a larger boost to accuracy
when the speaker was less reliable.

Next, we explored the factors that influenced inspection time on
exposure trials by predicting inspection times using reliability
condition and participants' use of the gaze cue as predictors. We found
a main effect of participants' use of the gaze cue (-0.32, \(p\)
\textless{} .001) with shorter inspection times when participants
followed gaze. The main effect of reliability condition and the
interaction between reliability and use of gaze were not significant.
These analyses provide evidence that inspection times were similar
across the different reliability conditions and that use of the gaze cue
was the primary factor affecting how long participants explored the
objects on exposure trials.

Together, these four analyses show that when the speaker's gaze was more
reliable, participants were more likely to: (a) use the gaze cue, (b)
rate the speaker as more reliable, and (c) store fewer word-object
links, showing behavior more consistent with single hypothesis tracking.
These findings support and extend the results of Experiments 1 and 2 in
several important ways. First, similar to Experiment 2, participants'
performance on Same trials was relatively unaffected by changes in
performance on Switch trials. The selective effect of gaze on Switch
trials provides converging evidence that the limitations on Same trials
may be different than those regulating the distribution of attention on
Switch trials. Second, learners' use of a referential cue was a stronger
predictor of reduced memory for alternative word-object links compared
to our reliability manipulation. Although we found a significant effect
of reliability on participants' use of the gaze cue, participants'
tendency to use the cue remained high. Consider that even in the 0\%
reliability condition the mean proportion of gaze following was still
0.82. It is reasonable that participants would continue to use the gaze
cue in our experiment since it was the only cue available and
participants did not have a strong reason to think that the speaker
would be deceptive.

The critical contribution of Experiment 3 is to show that learners
respond to a graded manipulation of referential uncertainty, with the
amount of information stored from the initial exposure tracking with the
reliability of the cue. This graded accuracy performance shows that
learners stored alternative word-object links with different levels of
fidelity depending on the amount of referential uncertainty present
during learning.

Across Experiments 1-3, learners tended to store fewer word-object links
in unambiguous learning contexts when a clear referential cue was
present. However, in all three experiments, participants' responses on
exposure trials controlled the length of the trial, meaning that when
participants used the gaze cue, they also spent less time visually
inspecting the objects. Thus, we do not know whether there is an
independent effect of referential cues the representations underlying
cross-situational learning, or if the effects found in Experiments 1-3
are entirely mediated by a reduction in inspection time. In Experiment
4, we addressed this possibility by removing participants' control over
the length of exposure trials, which made the inspection times
equivalent across the Gaze and No-Gaze conditions.

\section{Experiment 4}\label{experiment-4}

In Experiment 4, we asked whether a reduction in visual inspection time
in the gaze condition could completely explain the effect of social cues
on learners' reduced memory for alternative word-object links. To answer
this question, we modified our paradigm and made the length of exposure
trials equivalent across the Gaze and No-Gaze conditions. In this
version of the task, participants saw the objects for a fixed amount of
time regardless of whether gaze was present. We also included two
different exposure trial lengths in order to test whether gaze would
have a differential effect at shorter vs.~longer inspection times. If
the presence of gaze reduces learners' memory for multiple word-object
links, then this provides evidence that referential cues affected the
underlying representations over and above a reduction in inspection
time.

\begin{CodeChunk}
\begin{figure}[tb]

{\centering \includegraphics[width=0.5\linewidth]{figs/expt4-plot-1} 

}

\caption[Experiment 4 results]{Experiment 4 results. Accuracy on test trials in Experiment 4 collapsed across the Long and Short inspection time conditions. The dashed line represents chance performance. Color and line type indicate whether there was gaze present on exposure trials. Error bars indicate 95\% confidence intervals computed by non-parametric bootstrap. }\label{fig:expt4-plot}
\end{figure}
\end{CodeChunk}

\subsection{Method}\label{method-3}

\subsubsection{Participants}\label{participants-3}

Participant recruitment and inclusion/exclusion criteria were identical
to those of Experiments 1, 2, and 3. 100 HITs were posted for each
condition (1 Referent X 2 Intervals X 2 Inspection Time conditions) for
a total of 400 paid HITs (37 HITs excluded).

\subsubsection{Stimuli}\label{stimuli-2}

Audio, picture, and video stimuli were identical to Experiments 2 and 3.
Since inspection times were fixed across conditions, we wanted to ensure
that participants were aware of the time remaining on each exposure
trial. So we included a circular countdown timer located above the
center video. The timer remained on the screen during test trials but
did not count down since participants could take as much time as they
wanted to respond on test trials.

\subsubsection{Design and Procedure}\label{design-and-procedure-3}

Procedures were identical to those of Experiment 1-3. The design was
identical to that of Experiment 2 and consisted of 32 trials split into
2 blocks of 16 trials. Each block consisted of 8 exposure trials and 8
test trials (4 Same trials and 4 Switch trials) and contained only Gaze
or No-Gaze exposure trials. The order of block was counterbalanced
across participants.

The major design change was to make the length of exposure trials
equivalent across the Gaze and No-Gaze conditions. We randomly assigned
participants to one of two inspection time conditions: Short (6 seconds)
or Long (9 seconds). These times were selected based on participants'
self-paced inspection times in the Gaze and No-Gaze conditions in
Experiment 2. After pilot testing, we added three seconds to each
condition to ensure that participants had enough time to respond before
the experiment advanced. If participants did not respond in the allotted
time, an error message appeared informing participants that time had run
out and encouraged them to respond within the time window on subsequent
trials.

\subsection{Results and Discussion}\label{results-and-discussion-3}

We did not see strong evidence of an effect of the different inspection
times. Thus, all of the results reported here collapse across the short
and long inspection time conditions. For all analyses, we removed the
trials on which participants did not respond within the fixed inspection
time on exposure trials (0.05\% of trials).

\subsubsection{Exposure Trials}\label{exposure-trials-3}

Participants' responses on exposure trials differed from those expected
by chance (smallest \(\beta\) = 2.95, z = 38.08, \(p\) \textless{}
.001), suggesting that gaze was again effective in directing
participants' attention. Similar to Experiment 2, participants were
quite likely to use the gaze cue when it was a video of an actress
(\(M_{0-interval}\) = 0.93, \(M_{3-interval}\) = 0.95).

\subsubsection{Test Trials}\label{test-trials-3}

Figure 6 shows performance on test trials in Experiment 4. In the
majority of conditions, participants selected the correct referent at
rates greater than chance (smallest \(\beta\) = 0.2, z = 2.2, \(p\)
\textless{} .05). However, participants' responses were only marginally
different from chance on Switch trials after exposure trials with gaze
in the 3-interval condition (\(\beta\) = 0.17, \(p\) = 0.06).

We replicate the key finding from Experiments 1-3: after seeing exposure
trials with gaze, participants were less accurate on Switch trials
(\(\beta\) = 0.9, \(p\) \textless{} .001). Since inspection times were
fixed across the Gaze and No-Gaze conditions, this finding provides
evidence that the presence of a referential cue did more than just
reduce the amount of time participants' spent inspecting the potential
word-object links. In contrast to Experiments 2 and 3, visual inspection
of Figure 6 suggested that the referential cue provided a boost to
accuracy on Same trials. To assess the simple effect of gaze on trial
type, we computed pairwise contrasts using the \emph{lsmeans} package in
R with a Bonferroni correction for multiple comparisons (Lenth, 2016).
Accuracy was higher for Same trials in the Gaze condition (\(\beta\) =
0.49, \(p\) \textless{} .001), but lower for Switch trials (\(\beta\) =
-0.41, \(p\) \textless{} .001). The boost in accuracy on Same trials
differs from Experiments 2 and 3 and suggests that making inspection
times equivalent across conditions allowed the social cue to affect the
strength of learners' memory for their candidate hypothesis.

The results of Experiment 4 help to clarify the effect of gaze on memory
in our task, providing evidence that the presence of a referential cue
did more than just reduce participants' visual inspection time. Instead,
gaze reduced memory for alternative word-object links even when people
had the same opportunity to visually inspect and encode them. We also
found evidence of a boost for learners' memory of their candidate
hypothesis in the gaze condition, an effect that we saw at the higher
number of referents and the longer intervals in Experiment 1, but that
we did not see in Experiments 2 or 3. One explanation for this
difference is that in Experiment 4, since participants' use of gaze was
independent of the length of exposure trials, inspection times in the
gaze condition were longer compared to those in Experiments 1-3. Thus,
it could be that the combination of a gaze cue coupled with the
opportunity to continue attending to the gaze target led to a boost in
performance on Same trials relative to trials without gaze.

\section{General Discussion}\label{general-discussion}

Tracking cross-situational word-object statistics allows word learning
to proceed despite the presence of individually ambiguous naming events.
But models of cross-situational learning disagree about how much
information is actually stored in memory, and the input to statistical
learning mechanisms can vary along a continuum of referential
uncertainty from unambiguous naming instances to highly ambiguous
situations. In the current line of work, we explore the hypothesis that
these two factors are fundamentally linked to one another and to the
social context in which word learning occurs. Specifically, we ask how
cross-situational learning operates over social input that varies the
amount of ambiguity in the learning context.

Our results suggest that the representations underlying
cross-situational learning are quite flexible. In the absence of a
referential cue to word meaning, learners tended to store more
alternative word-object links. In contrast, when gaze was present
learners stored less information, showing behavior consistent with
tracking a single hypothesis (Experiments 1 and 2). Learners were also
sensitive to a parametric manipulation of the strength of the
referential cue, showing a graded increase in the tendency to use the
cue as reliability increased, which in turn resulted in a graded
decrease in memory for alternative word-object links (Experiment 3).
Finally, learners stored less information in the presence of gaze even
when they spent the same amount of time visually inspecting the objects
during learning (Experiment 4).

In Experiments 2 and 3 reduced memory for alternative hypotheses did not
result in a boost to memory for learners' candidate hypothesis. This
pattern of data suggests that the presence of a referential cue
selectively affected one component of the underlying representation: the
number of alternative word-object links, and not the strength learners'
candidate hypothesis. However, in Experiments 1 and 4, we did see some
evidence of stronger memory for learners' initial hypothesis in the
presence of gaze: at the higher number of referents and interval
conditions (Experiment 1), and when the length of exposure trials was
equivalent across the Gaze and No-Gaze conditions (Experiment 4). We
speculate that the relationship between the presence of a referential
cue and the strength of learners' candidate hypothesis is modulated by
how the cue interacts with attention. In Experiment 1, gaze may have
provided a boost because, in the absence of gaze, attention would have
been distributed across a larger number of alternatives. And, in
Experiment 4, gaze may have led to better memory because it was coupled
with the opportunity for sustained attention to the gaze target. More
work is needed in order to understand precisely when the presence of
gaze affects this particular component of the representations underlying
cross-situational learning.

In Experiments 1-3, longer inspection times (i.e., more time spent
encoding the word-object links during learning) led to better memory at
test. We did, however, find slightly different interaction effects
across our studies. In Experiment 1, longer inspection times led to
higher accuracy in the No-Gaze condition for both Same and Switch
trials. In Experiment 2, longer inspection times provided a larger boost
to performance on Switch trials compared to Same trials, regardless of
gaze condition. And in Experiment 3, we found some evidence that longer
inspection times led to better memory when the gaze cue was less
reliable. Despite these differences, we speculate that inspection time
played a similar role across these studies: When a social cue was
present, learners' attention was focused and inspection times tended to
be shorter, which led to worse performance on Switch trials (i.e.,
reduced memory for alternative word-object links). Interestingly, in
Experiment 4, we found an effect of social cues on memory for
alternatives even when inspection times were equivalent, suggesting that
gaze does more than just modulate visual attention during learning.

\subsection{Relationship to previous
work}\label{relationship-to-previous-work}

Why might a decrease in memory for alternatives fail to increase the
strength of learners' memory for their candidate hypothesis? One
possibility is that participants did not shift their cognitive resources
from the set of alternatives to their single hypothesis, but instead
chose to use the gaze information to reduce inspection time, thus
conserving their resources for future use. Griffiths, Lieder, and
Goodman (2015) formalize this behavior by pushing the rationality of
computational-level models down to the psychological process level. In
their framework, cognitive systems are thought to be adaptive in that
they optimize the use of their limited resources, taking the cost of
computation (e.g., the opportunity cost of time or mental energy) into
account. For example, Vul, Goodman, Griffiths, and Tenenbaum (2014)
showed that as time pressure increased in a decision-making task,
participants were more likely to show behavior consistent with a less
cognitively challenging strategy of matching, rather than with the
globally optimal strategy. In the current work, we found that learners
showed evidence of altering how they allocated cognitive resources based
on the amount of referential uncertainty present during learning,
spending less time inspecting alternative word-object links and reducing
the number of links stored in memory when uncertainty was low.

Our results fit well with recent experimental work that investigates how
attention and memory can constrain infants' statistical word learning.
For example, Smith and Yu (2013) used a modified cross-situational
learning task to show that only infants who disengaged from a novel
object to look at both potential referents were able to learn the
correct word--object mappings. Moreover, Vlach and Johnson (2013) showed
that 16-month-olds were only able to learn from adjacent
cross-situational co-occurrence statistics, and unable to learn from
co-occurrences that were separated in time. Both of these findings make
the important point that only the information that comes into contact
with the learning system can be used for cross-situational word
learning, and this information is directly influenced by the attention
and memory constraints of the learner. These results also add to a large
literature showing the importance of social information for word
learning (P. Bloom, 2002; Clark, 2009) and to recent work exploring the
interaction between statistical learning mechanisms and other types of
information (Frank, Goodman, \& Tenenbaum, 2009; Koehne \& Crocker,
2014; Yu \& Ballard, 2007). Our findings suggest that referential cues
affect statistical learning by modulating the amount of information that
learners store in the underlying representations that support learning
over time.

Is gaze a privileged cue, or could other, less-social cues (e.g., an
arrow) also affect the representations underlying cross-situational
learning? On the one hand, previous research has shown that gaze cues
lead to more reflexive attentional responses compared to arrows
(Friesen, Ristic, \& Kingstone, 2004), that gaze-triggered attention
results in better learning compared to salience-triggered attention (Wu
\& Kirkham, 2010), and that even toddlers readily use gaze to infer
novel word meanings (Baldwin, 1993). Thus, it could be that gaze is an
especially effective cue for constraining word learning since it
communicates a speaker's referential intent and is a particularly good
way to guide attention. On the other hand, the generative process of the
cue -- whether it is more or less social in nature -- might be less
important; instead, the critical factor might be whether the cue
effectively reduces uncertainty in the naming event. Under this account,
gaze is placed amongst a set of many cues that could produce similar
effects as those reported here. Future work could explore a wider range
of cues to see if they modulate the representations underlying
cross-situational learning in a similar way.

How should we characterize the effect of gaze on attention and memory in
our task? One possibility is that the referential cue acts as a filter,
only allowing likely referents to contact statistical learning
mechanisms (Yu \& Ballard, 2007). This `filtering account' separates the
effect of social cues from the underlying computation that aggregates
cross-situational information. Another possibility is that referential
cues provide evidence about a speaker's communicative intent (Frank et
al., 2009). In this model, the learner is reasoning about the speaker
and word meanings simultaneously, which places inferences based on
social information as part of the underlying computation. A third
possibility is that participants thought of the referential cue as
pedagogical. In this context, learners assume that the speaker will
choose an action that is most likely to increase the learner's belief in
the true state of the world (Shafto, Goodman, \& Frank, 2012), making it
unnecessary to allocate resources to alternative hypotheses. Experiments
show that children spend less time exploring an object and are less
likely to discover alternative object-functions if a single function is
demonstrated in a pedagogical context (Bonawitz et al., 2011). However,
because the results from the current study cannot distinguish between
these explanations, these questions remain topics for future studies
specifically designed to tease apart these possibilities.

\subsection{Limitations}\label{limitations}

There are several limitations to the current study that are worth
noting. First, the social context that we used was relatively
impoverished. Although we moved beyond a simple manipulation of the
presence or absence of social information in Experiment 3, we
nevertheless isolated just a single cue to reference, gaze. But
real-world learning contexts are much more complex, providing learners
access to multiple cues such as gaze, pointing, and previous discourse.
In fact, Frank, Tenenbaum, and Fernald (2013) analyzed a corpus of
parent-child interactions and concluded that learners would do better to
aggregate noisy social information from multiple cues, rather than
monitor a single cue since no single cue was a consistent predictor of
reference. In our data, we did see a more reliable effect of referential
cues when we used video of an actress, which included both gaze and head
turn as opposed to the static, schematic stimuli, which only included
gaze. It is still an open and interesting question as to how our results
would generalize to learning environments that contain a rich
combination of social cues.

Second, we do not yet know how variations in referential uncertainty
during learning would affect the representations of young word learners,
the age at which cross-situational word learning might be particularly
important. Recent research using a similar paradigm as our own did not
find evidence that 2- or 3-year-olds stored multiple word-object links;
instead, children only retained a single candidate hypothesis (Woodard,
Gleitman, \& Trueswell, 2016). However, performance limitations on
children's developing attention and memory systems (Colombo, 2001;
Ross-sheehy, Oakes, \& Luck, 2003) could make success on these explicit
response tasks more difficult. Moreover, our work suggests that
different levels of referential uncertainty in naturalistic learning
contexts (see Medina, Snedeker, Trueswell, \& Gleitman, 2011; Yurovsky
\& Frank, 2015) might evoke different strategies for information
storage, with learners retaining more information as ambiguity in the
input increases. Thus, we think that it will be important to test a
variety of outcome measures and learning contexts to see if younger
learners show evidence of storing multiple word meanings during
learning.

In addition, previous work with infants has shown that their attention
is often stimulus-driven and sticky (Oakes, 2011), suggesting that very
young word learners might not effectively explore the visual scene in
order to extract the necessary statistics for storing multiple
alternatives. It could be that referential cues play an even more
important role for young learners by filtering the input to
cross-situational word learning mechanisms and guiding children to the
relevant statistics in the input. In fact, recent work has shown that
the precise timing of features such as increased parent attention and
gesturing towards a named object and away from non-target objects were
strong predictors of referential clarity in a naming event (Trueswell et
al., 2016). It could be that the statistics available in these
particularly unambiguous naming events are the most useful for
cross-situational learning.

Finally, the current experiments used a restricted cross-situational
word learning scenario, which differs from real-world language learning
contexts in several important ways. One, we only tested a single
exposure for each novel word-object pairing; whereas, real-world naming
events are best characterized by discourse where an object is likely to
be named repeatedly in a short amount of time (Frank, Tenenbaum, \&
Fernald, 2013; Rohde \& Frank, 2014). Two, the restricted visual world
of 2-8 objects on a screen combined with the forced-choice response
format may have biased people to assume that all words in the task must
have referred to one of the objects. But, in actual language use, people
can refer to things that are not physically co-present (e.g., Gleitman,
1990), creating a scenario where learners would not benefit from storing
additional word-object links in the absence of clear referential cues.
Finally, we presented novel words in isolation, removing any sentential
cues to word meaning (e.g., verb-argument relations). In fact, previous
work with adults has shown that cross-situational learning mechanisms
only operate in contexts where sentence-level constraints do not
completely disambiguate meaning (Koehne \& Crocker, 2014). Thus, we need
more evidence to understand how the representations underlying
cross-situational learning change in response to referential uncertainty
at different timescales and in richer language contexts that more
accurately reflect real-world learning environments.

\subsection{Conclusions}\label{conclusions}

Word learning proceeds despite the potential for high levels of
referential uncertainty and despite learners' limited cognitive
resources. Our work shows that cross-situational learners flexibly
respond to the amount of ambiguity in the input, and as referential
uncertainty increases, learners tended to store more word-object links.
Overall, these results bring together aspects of social and statistical
accounts of word learning to increase our understanding of how
statistical learning mechanisms operate over fundamentally social input.

\newpage

\section{Acknowledgements}\label{acknowledgements}

We are grateful to Rose Schneider for helping record stimuli and to the
members of the Language and Cognition Lab for their feedback on this
project. This work was supported by a National Science Foundation
Graduate Research Fellowship to KM, an NIH NRSA Postdoctoral Fellowship
to DY, and a John Merck Scholars Fellowship to M.C.F.

\newpage

\section{References}\label{references}

\setlength{\parindent}{-0.1in} \setlength{\leftskip}{0.125in} \noindent

\hypertarget{refs}{}
\hypertarget{ref-baldwin1993infants}{}
Baldwin, D. A. (1993). Infants' ability to consult the speaker for clues
to word reference. \emph{Journal of Child Language}, \emph{20}(02),
395--418.

\hypertarget{ref-barr2013random}{}
Barr, D. J. (2013). Random effects structure for testing interactions in
linear mixed-effects models. \emph{Frontiers in Psychology}, \emph{4},
328.

\hypertarget{ref-bates2013lme4}{}
Bates, D., Maechler, M., Bolker, B., \& Walker, S. (2013). Lme4: Linear
mixed-effects models using eigen and s4. \emph{R Package Version},
\emph{1}(4).

\hypertarget{ref-bloom2002children}{}
Bloom, P. (2002). \emph{How children learn the meaning of words}. The
MIT Press.

\hypertarget{ref-bonawitz2011double}{}
Bonawitz, E., Shafto, P., Gweon, H., Goodman, N. D., Spelke, E., \&
Schulz, L. (2011). The double-edged sword of pedagogy: Instruction
limits spontaneous exploration and discovery. \emph{Cognition},
\emph{120}(3), 322--330.

\hypertarget{ref-brooks2005development}{}
Brooks, R., \& Meltzoff, A. N. (2005). The development of gaze following
and its relation to language. \emph{Developmental Science}, \emph{8}(6),
535--543.

\hypertarget{ref-brooks2008infant}{}
Brooks, R., \& Meltzoff, A. N. (2008). Infant gaze following and
pointing predict accelerated vocabulary growth through two years of age:
A longitudinal, growth curve modeling study. \emph{Journal of Child
Language}, \emph{35}(01), 207--220.

\hypertarget{ref-carpenter1998social}{}
Carpenter, M., Nagell, K., Tomasello, M., Butterworth, G., \& Moore, C.
(1998). Social cognition, joint attention, and communicative competence
from 9 to 15 months of age. \emph{Monographs of the Society for Research
in Child Development}, i--174.

\hypertarget{ref-cartmill2013quality}{}
Cartmill, E. A., Armstrong, B. F., Gleitman, L. R., Goldin-Meadow, S.,
Medina, T. N., \& Trueswell, J. C. (2013). Quality of early parent input
predicts child vocabulary 3 years later. \emph{Proceedings of the
National Academy of Sciences}, \emph{110}(28), 11278--11283.

\hypertarget{ref-clark2009first}{}
Clark, E. V. (2009). \emph{First language acquisition}. Cambridge
University Press.

\hypertarget{ref-cleveland2007joint}{}
Cleveland, A., Schug, M., \& Striano, T. (2007). Joint attention and
object learning in 5-and 7-month-old infants. \emph{Infant and Child
Development}, \emph{16}(3), 295--306.

\hypertarget{ref-colombo2001development}{}
Colombo, J. (2001). The development of visual attention in infancy.
\emph{Annual Review of Psychology}, \emph{52}(1), 337--367.

\hypertarget{ref-frank2009using}{}
Frank, M. C., Goodman, N. D., \& Tenenbaum, J. B. (2009). Using
speakers' referential intentions to model early cross-situational word
learning. \emph{Psychological Science}, \emph{20}(5), 578--585.

\hypertarget{ref-frank2013social}{}
Frank, M. C., Tenenbaum, J. B., \& Fernald, A. (2013). Social and
discourse contributions to the determination of reference in
cross-situational word learning. \emph{Language Learning and
Development}, \emph{9}(1), 1--24.

\hypertarget{ref-friesen2004attentional}{}
Friesen, C. K., Ristic, J., \& Kingstone, A. (2004). Attentional effects
of counterpredictive gaze and arrow cues. \emph{Journal of Experimental
Psychology: Human Perception and Performance}, \emph{30}(2), 319.

\hypertarget{ref-gillette1999human}{}
Gillette, J., Gleitman, H., Gleitman, L., \& Lederer, A. (1999). Human
simulations of vocabulary learning. \emph{Cognition}, \emph{73}(2),
135--176.

\hypertarget{ref-gleitman1990structural}{}
Gleitman, L. (1990). The structural sources of verb meanings.
\emph{Language Acquisition}, \emph{1}(1), 3--55.

\hypertarget{ref-griffiths2015rational}{}
Griffiths, T. L., Lieder, F., \& Goodman, N. D. (2015). Rational use of
cognitive resources: Levels of analysis between the computational and
the algorithmic. \emph{Topics in Cognitive Science}, \emph{7}(2),
217--229.

\hypertarget{ref-hollich2000breaking}{}
Hollich, G. J., Hirsh-Pasek, K., Golinkoff, R. M., Brand, R. J., Brown,
E., Chung, H. L., \ldots{} Bloom, L. (2000). Breaking the language
barrier: An emergentist coalition model for the origins of word
learning. \emph{Monographs of the Society for Research in Child
Development}, i--135.

\hypertarget{ref-kanwisher1997locus}{}
Kanwisher, N., Woods, R. P., Iacoboni, M., \& Mazziotta, J. C. (1997). A
locus in human extrastriate cortex for visual shape analysis.
\emph{Journal of Cognitive Neuroscience}, \emph{9}(1), 133--142.

\hypertarget{ref-koehne2014interplay}{}
Koehne, J., \& Crocker, M. W. (2014). The interplay of cross-situational
word learning and sentence-level constraints. \emph{Cognitive Science}.

\hypertarget{ref-koenig2004trust}{}
Koenig, M. A., Clement, F., \& Harris, P. L. (2004). Trust in testimony:
Children's use of true and false statements. \emph{Psychological
Science}, \emph{15}(10), 694--698.

\hypertarget{ref-lenth2016lsmeans}{}
Lenth, R. V. (2016). Least-squares means: The R package lsmeans.
\emph{Journal of Statistical Software}, \emph{69}(1), 1--33.
\url{http://doi.org/10.18637/jss.v069.i01}

\hypertarget{ref-mcmurray2012word}{}
McMurray, B., Horst, J. S., \& Samuelson, L. K. (2012). Word learning
emerges from the interaction of online referent selection and slow
associative learning. \emph{Psychological Review}, \emph{119}(4), 831.

\hypertarget{ref-medina2011words}{}
Medina, T. N., Snedeker, J., Trueswell, J. C., \& Gleitman, L. R.
(2011). How words can and cannot be learned by observation.
\emph{Proceedings of the National Academy of Sciences}, \emph{108}(22),
9014--9019.

\hypertarget{ref-oakes2011infant}{}
Oakes, L. M. (2011). \emph{Infant perception and cognition: Recent
advances, emerging theories, and future directions}. Oxford University
Press, USA.

\hypertarget{ref-quine19600}{}
Quine, W. V. (1960). 0. word and object. \emph{111e MIT Press}.

\hypertarget{ref-rohde2014markers}{}
Rohde, H., \& Frank, M. C. (2014). Markers of topical discourse in
child-directed speech. \emph{Cognitive Science}, \emph{38}(8),
1634--1661.

\hypertarget{ref-ross2003development}{}
Ross-sheehy, S., Oakes, L. M., \& Luck, S. J. (2003). The development of
visual short-term memory capacity in infants. \emph{Child Development},
\emph{74}(6), 1807--1822.

\hypertarget{ref-shafto2012learning}{}
Shafto, P., Goodman, N. D., \& Frank, M. C. (2012). Learning from others
the consequences of psychological reasoning for human learning.
\emph{Perspectives on Psychological Science}, \emph{7}(4), 341--351.

\hypertarget{ref-siskind1996computational}{}
Siskind, J. M. (1996). A computational study of cross-situational
techniques for learning word-to-meaning mappings. \emph{Cognition},
\emph{61}(1), 39--91.

\hypertarget{ref-smith2011cross}{}
Smith, K., Smith, A. D., \& Blythe, R. A. (2011). Cross-situational
learning: An experimental study of word-learning mechanisms.
\emph{Cognitive Science}, \emph{35}(3), 480--498.

\hypertarget{ref-smith2008infants}{}
Smith, L. B., \& Yu, C. (2008). Infants rapidly learn word-referent
mappings via cross-situational statistics. \emph{Cognition},
\emph{106}(3), 1558--1568.

\hypertarget{ref-smith2013visual}{}
Smith, L. B., \& Yu, C. (2013). Visual attention is not enough:
Individual differences in statistical word-referent learning in infants.
\emph{Language Learning and Development}, \emph{9}(1), 25--49.

\hypertarget{ref-smith2014unrealized}{}
Smith, L. B., Suanda, S. H., \& Yu, C. (2014). The unrealized promise of
infant statistical word--referent learning. \emph{Trends in Cognitive
Sciences}, \emph{18}(5), 251--258.

\hypertarget{ref-trueswell2016perceiving}{}
Trueswell, J. C., Lin, Y., Armstrong, B., Cartmill, E. A.,
Goldin-Meadow, S., \& Gleitman, L. R. (2016). Perceiving referential
intent: Dynamics of reference in natural parent--child interactions.
\emph{Cognition}, \emph{148}, 117--135.

\hypertarget{ref-trueswell2013propose}{}
Trueswell, J. C., Medina, T. N., Hafri, A., \& Gleitman, L. (2013).
Propose but verify: Fast mapping meets cross-situational word learning.
\emph{Cognitive Psychology}, \emph{66}(1), 126--156.

\hypertarget{ref-vlach2013memory}{}
Vlach, H. A., \& Johnson, S. P. (2013). Memory constraints on infants'
cross-situational statistical learning. \emph{Cognition}, \emph{127}(3),
375--382.

\hypertarget{ref-vouloumanos2008fine}{}
Vouloumanos, A. (2008). Fine-grained sensitivity to statistical
information in adult word learning. \emph{Cognition}, \emph{107}(2),
729--742.

\hypertarget{ref-vul2014}{}
Vul, E., Goodman, N., Griffiths, T. L., \& Tenenbaum, J. B. (2014). One
and done? Optimal decisions from very few samples. \emph{Cognitive
Science}, \emph{38}(4), 599--637.

\hypertarget{ref-woodard2016two}{}
Woodard, K., Gleitman, L. R., \& Trueswell, J. C. (2016). Two-and
three-year-olds track a single meaning during word learning: Evidence
for propose-but-verify. \emph{Language Learning and Development},
\emph{12}(3), 252--261.

\hypertarget{ref-wu2010no}{}
Wu, R., \& Kirkham, N. Z. (2010). No two cues are alike: Depth of
learning during infancy is dependent on what orients attention.
\emph{Journal of Experimental Child Psychology}, \emph{107}(2),
118--136.

\hypertarget{ref-wu2011infants}{}
Wu, R., Gopnik, A., Richardson, D. C., \& Kirkham, N. Z. (2011). Infants
learn about objects from statistics and people. \emph{Developmental
Psychology}, \emph{47}(5), 1220.

\hypertarget{ref-yoon2008communication}{}
Yoon, J. M., Johnson, M. H., \& Csibra, G. (2008). Communication-induced
memory biases in preverbal infants. \emph{Proceedings of the National
Academy of Sciences}, \emph{105}(36), 13690--13695.

\hypertarget{ref-yoshida2012exclusion}{}
Yoshida, K., Rhemtulla, M., \& Vouloumanos, A. (2012). Exclusion
constraints facilitate statistical word learning. \emph{Cognitive
Science}, \emph{36}(5), 933--947.

\hypertarget{ref-yu2007unified}{}
Yu, C., \& Ballard, D. H. (2007). A unified model of early word
learning: Integrating statistical and social cues.
\emph{Neurocomputing}, \emph{70}(13), 2149--2165.

\hypertarget{ref-yu2007rapid}{}
Yu, C., \& Smith, L. B. (2007). Rapid word learning under uncertainty
via cross-situational statistics. \emph{Psychological Science},
\emph{18}(5), 414--420.

\hypertarget{ref-yu2012embodied}{}
Yu, C., \& Smith, L. B. (2012). Embodied attention and word learning by
toddlers. \emph{Cognition}.

\hypertarget{ref-yurovsky2014algorithmic}{}
Yurovsky, D., \& Frank, M. C. (2015). An integrative account of
constraints on cross-situational learning. \emph{Cognition}.

\hypertarget{ref-yurovsky2013statistical}{}
Yurovsky, D., Smith, L. B., \& Yu, C. (2013). Statistical word learning
at scale: The baby's view is better. \emph{Developmental Science},
\emph{16}(6), 959--966.

\bibliography{}

\end{document}
